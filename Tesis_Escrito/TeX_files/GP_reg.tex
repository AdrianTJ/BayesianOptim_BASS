\chapter{Gaussian Process Regression}

One of the applications of Gaussian processes is to use them as a method to perform supervised learning using it. The method generates from data a mean function and then uses that mean function to extrapolate or predict on points where there is no data observed. As we described in the last chapter, we can fully specify or determine a particular Gaussian process by specifying its mean $m(x) = \mathbb{E}[f(x)]$ and its kernel function $k(x,x')$. While in the last chapter we started with the Gaussian process and from that generate the values on the rest of the input space, in this chapter we are going to be doing somewhat of the inverse of that. We will be starting from a set of data $\{ x_i \}_{i \leq n} = X \subset \mathcal{X}$, and adjusting the properties of the kernel and mean to achieve the best fit Gaussian process to the points and minimize the error. 